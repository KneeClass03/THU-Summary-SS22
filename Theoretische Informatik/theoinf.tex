\documentclass{scrreprt}

\usepackage{graphicx}
\usepackage{tikz}
\usepackage{amstext}
\usepackage{hyperref}
\usepackage{amsfonts}
\usepackage{tabularx}
\usepackage{multirow}

\hypersetup{
  colorlinks=false,
  linkbordercolor=blue,
  pdfborderstyle={/S/U/W 0}
}
\title{\textbf{Theoretische Informatik}}
\author{\href{https://github.com/bircni}{\color{black}github/bircni}}
\date{}
\begin{document}
\pagenumbering{gobble}  
\maketitle
\pagebreak
\renewcommand{\contentsname}{Inhaltsverzeichnis}
%\renewcommand{\cftdot}{}
\setcounter{tocdepth}{1}
\tableofcontents
\addtocontents{toc}{~\hfill\textbf{Seite}\par}
\pagebreak
\pagenumbering{arabic}
\chapter{Mathematische Grundbegriffe}
\section{Mengen}
\begin{itemize}
    \item Eigenschaften von Mengen
    \\Eine Menge ist eine Sammlung von Elementen, diese können alles mögliche sein.
    \\Wesentliche Eigenschaften von Mengen:
    \begin{itemize}
        \item Mengen haben keine Anordnungsreihenfolge
        \item Ein Element kann höchstens einmal in einer Menge enthalten sein
        \\Bsp. ${a,a,b}={a,b}$
    \end{itemize}
    \item Grundmengen von Zahlen
    \begin{itemize}
        \item $\mathbb{N}$: Natürliche Zahlen
        \item $\mathbb{Z}$: Ganze Zahlen
        \item $\mathbb{Q}$: Rationale Zahlen
        \item $\mathbb{R}$: Reelle Zahlen
    \end{itemize}
    \item Mengenoperationen
    \begin{itemize}
        \item $\emptyset$: leere Menge
        \item $x \in A$: x ist Element der Menge A
        \item $|A|$: Kardinalität der Menge, bzw. Anzahl der Elemente
        \item $A \subseteq B$: A ist Teilmenge von B oder gleich der Menge von B
        \item $A \subset B$: Menge A ist echt in Menge B enthalten
        \item $A \cup B$: \textbf{Vereinigung:} Menge aller Elemente in A oder B
        \item $A \cap B$: \textbf{Schnittmenge:} Menge aller Elemente, sowohl in A als auch in B
        \item $A \backslash B$: \textbf{Differenzmenge:} alle Elemente, die in A aber nicht in B enthalten sind
        \item $A \times B$: \textbf{kartesisches Produkt:} Menge aller Paare, die aus A und B gebildet werden können
        \item $\mathcal{A}$: \textbf{Potenzmenge:} Menge aller Teilmengen von A
    \end{itemize}
    \item Kardinalität der Potenzmenge
    \\Für die Potenzmenge $\mathcal{M}$  einer endlichen Menge M gilt: $|\mathcal{M}| = 2^{|M|}$
\end{itemize}
\section{Relationen}
Eine Relation ist eine Beziehung zwischen Elementen einer Menge.
\\Eine zweistellige Relation R über einer Grundmenge M ist eine Menge von Paaren (x,y)
mit $x \in M$ und $y \in M$, d.h. $R \subseteq M \times M$
\section{Funktionen}
\begin{itemize}
    \item Eine Funktion $f: A \rightarrow B$ ist eine zweistellige Relation $f \subseteq A \times B$ mit der Eigenschaft:
    \\$(a,b_{1}) \in f \and (a,b_{2}) \in f \Rightarrow b_{1}=b_{2}$
    \item Ist $(a,b) \in f$, dann heißt b Funktionswert zu a, $f(a)=b$
    \item Eine Funktion $f:A \rightarrow B$ heißt \textbf{total}, wenn es für jedes Argument $a \in A$ einen Funktionswert
    $f(a)=b \in B$ gibt. Sonst heißt die Funktion \textbf{partiell}.
    \item Gibt es für ein $a \in A$ keinen Funktionswert , dann ist f an dieser Stelle \textbf{undefiniert} $f(a)=\perp$
\end{itemize}
\section{Unendliche Mengen}
\begin{itemize}
    \item \textbf{abzählbar unendlich}: Gleich mächtig, wie die Menge der natürlichen Zahlen.
    \\Also: für jedes Element gibt es eine Position und für jede Position i gibt es auch einen Wert $m_{i} \in M$
    \item \textbf{überabzählbar unendlich}: unendlich und nicht abzählbare Menge
    \\Keine 1-zu-1-Zuordnung zu den natürlichen Zahlen
    \\Bsp: Die Menge der unendlich langen 0/1 Folgen, die Potenzmenge der natürlichen Zahlen
\end{itemize}
\pagebreak
\chapter{Grundbegriffe der Graphentheorie}
\section{Gerichtete Graphen}
\subsection{Definition}
\begin{itemize}
    \item Ein gerichteter Graph $G = (V,E)$ besteht aus
    \\V \hspace{1,5cm} Menge von Knoten
    \\$E \subseteq V \times V$ Menge von Kanten
    \item Für eine Kante $e = (u,v)$ ist u der Ausgangs- und v der Zielknoten
    \item Existitert eine Kante $e = (u,v)$, dann ist v ein Nachbar von u. Sie sind adjazent.
    \item Eine Kante mit gleichem Ausgangs- und Zielknoten heißt Schlinge/Schleife.
\end{itemize}
\subsection{Diagrammdarstellung von Graphen}
\begin{itemize}
    \item Knoten werden als Kreise dargestellt.
    \item Kanten werden als Pfeil vom Ausgangs- zum Zielknoten dargestellt.
    \item Die Positionierung von Knoten ist irrelevant, Kanten müssen keine geraden Linien sein.
\end{itemize}
\subsection{Weitere Darstellungsmöglichkeiten von Graphen}
Graphen können als Adjazenzmatrix oder in Adjazenzlisten dargestellt werden:\\
\includegraphics[width=0.8\textwidth]{"graphics/adjazenz"}
\subsection{Knotengrad}
Für einen Knoten eines gerichteten Graphen $G=(V,E)$ ist
\begin{itemize}
    \item der \textbf{Eingangsgrad} die Anzahl der Zielknoten v
    \item der \textbf{Ausgangsgrad} die Anzahl der Ausgangsknoten v
    \item der \textbf{Grad} die Summe von Ausgangsgrad und Eingangsgrad von v
\end{itemize}
\section{Ungerichtete Graphen}
\subsection{Definition}
\begin{itemize}
    \item Ein ungerichteter Graph $G=(V,E)$ besteht aus einer Knotenmenge V und einer Kantenmenge E
    \item Der Grad eines Knotens ist die Anzahl der von v ausgehenden Kanten.
    \\(Schlingen werden doppelt gezählt)
\end{itemize}
\subsection{Vollständige Graphen}
Ein ungerichteter Graph heißt \textbf{vollständiger Graph}, wenn es zw. je zwei verschiedenen Knoten eine Kante gibt.\\
\includegraphics[width=0.8\textwidth]{"graphics/vollst-Graph"}
\\
Ein vollständiger Graph mit \textbf{n} Knoten hat $\frac{n(n-1)}{2}$ Kanten.
\subsection{Bipartite Graphen}
\includegraphics[width=1\textwidth]{"graphics/bipartit"}
\pagebreak
\subsection{Planare Graphen}
\textbf{Definition:}
\\Ein Graph G heißt \textbf{planar}, wenn er in einer Ebene so gezeichnet werden kann, dass sich keine Kanten kreuzen.
\\\textbf{Kuratowski-Graphen}
\\\includegraphics[width=1\textwidth]{"graphics/kuratowski"}
\pagebreak
\section{Wege, Zyklen und Kreise}
\subsection{Definition}
\begin{itemize}
    \item Ein \textbf{Zyklus} ist ein Weg der Länge n>0 - Ausgangs und Endknoten stimmen überein
    \item Ein \textbf{Kreis} ist ein Zyklus, bei dem kein Knoten doppelt besucht wird
    \item Ein gerichteter Graph heißt \textbf{azyklischer Graph}, wenn er keinen Zyklus enthält
\end{itemize}
\section{Zusammenhang}
\subsection{Zusammenhang bei ungerichteten Graphen}
Ein ungerichteter Graph heißt \textbf{zusammenhängend}, wenn es von jedem Knoten zu jedem anderen mind. einen Weg gibt.
\\Ein Teilgraph des ungerichteten Graphen heißt \textbf{Zusammenhangskomponente}, wenn folgende Bedingungen gelten:
\begin{itemize}
    \item G' ist zusammenhängend
    \item G hat keinen größeren Teilgraphen G'', der zusammenhängend ist und G' als Teilgraph enthält 
\end{itemize}
Die Knotenmenge von G' ist eine Teilmenge der Knotenmenge von G
\subsection{Zusammenhang bei gerichteten Graphen}
\textbf{stark zusammenhängend}: für jede Kombination von $v_{1},v_{2}$ gibt es jeweils einen Weg
\\\textbf{(schwach) zusammenhängend}: wenn er als ungerichteter Graph zusammenhängend wäre
\section{Euler-Zyklen und -Wege}
\subsection{Definition}
\begin{itemize}
    \item Ein \textbf{Euler-Weg} für G ist ein Weg in dem jede Kante genau einmal durchlaufen wird
    \item Ein \textbf{Euler-Zyklus} für G ist ein Zyklus in dem jede Kante genau einmal durchlaufen wird
    \item G hat einen \textbf{Euler-Zyklus} genau dann, wenn alle Knoten einen \textbf{geraden Grad} haben
    \item G hat einen \textbf{Euler-Weg}, wenn ganu zwei Knoten ungeraden Grad haben und alle anderen einen geraden
\end{itemize}
\section{Hamilton-Kreise}
Ein Zyklus, bei dem jeder Knoten genau einmal besucht wird, nennt man \textbf{Hamilton-Kreis}.
\section{Bäume}
\subsection{Definition}
Ein Baum ist ein ungerichterter, zusammenhängender Graph ohne Schlingen.
\\Alle Knoten mit Grad 1 nennt man \textbf{Blätter} des Baums, die anderen heißen \textbf{innere Knoten}
\\Jeder nicht leere Baum mit \textbf{n Knoten} hat \textbf{n-1 Kanten}
\subsection{Spannbäume}
Ein Spannbaum ist ein Teilgraph eines ungerichteten Graphen, der ein Baum ist und alle Knoten dieses Graphen enthält.
\\Spannbäume existieren nur in zusammenhängenden Graphen.
\subsection{Wurzelbäume}
Ein gewurzelter Baum hat einen ausgezeichneten Knoten als \textbf{Wurzel}
\\Jeder Knoten kann beliebig viele Nachfolger haben, diese nennt man Kinder
\\Jeder Knoten hat genau einen Elternknoten
\\\textbf{Blätter} sind Knoten ohne Kinder, \textbf{innere Knoten} sind Knoten mit Kindern
\\\textbf{Definition}:\\
Ein gerichteter, zusammenhängender, azyklischer Graph ist ein Wurzelbaum, wenn
\begin{itemize}
    \item es genau einen Knoten w mit Eingangsgrad 0 gibt (Wurzel)
    \item alle anderen Knoten den Eingangsgrad 1 haben
\end{itemize}
\subsection{Tiefe von Knoten und Höhe von Bäumen}
Die \textbf{Tiefe} eines Knoten ist die Länge des Pfades vom Wurzelknoten zum Knoten k
\\Ein leerer Baum hat die Höhe h=0
\\Ein Baum aus einem einzigen Knoten hat die Höhe h=1\\
\includegraphics[width=1\textwidth]{"graphics/hoehe"}
\section{Binärbäume}
Ein Binärbaum ist ein Wurzelbaum, dessen Knoten maximal den Ausgangsgrad 2 haben,
d.h. ein Knoten hat maximal 2 Kinder
\subsection{Vollständige Binärbäume}
Ein Binärbaum heißt \textbf{vollständig}, wenn
\begin{itemize}
    \item alle innere Knoten den Ausgangsgrad 2 haben
    \item alle Blätter die gleiche Tiefe haben
\end{itemize}
\includegraphics[width=0.6\textwidth]{"graphics/vollBin"}
\\\textbf{Eigenschaften:}
\\Ein vollständiger Binärbaum der Höhe $h>0$ hat
\begin{itemize}
    \item \textbf{$2^{(h-1)}$ Blätter}
    \item \textbf{$2^{h}-1$ Knoten}
\end{itemize}
\subsection{Traversierung von Binärbäumen}
\begin{itemize}
    \item Preorder-Traversierung (W-L-R)
    \item Inorder-Traversierung (L-W-R)
    \item Postorder-Traversierung (L-R-W)
\end{itemize}
\pagebreak
\chapter{Formale Sprachen}
\section{Alphabete und Wörter}
Ein \textbf{Alphabet $\sum$} ist eine endliche, nicht leere Menge von Zeichen.
\subsection{Wörter}
\textbf{Definition}
\begin{itemize}
    \item Ein Wort über einem Alphabet ist eine endlich lange Folge von Zeichen aus $\sum$
    \item $|w|$ bezeichnet die Länge des Worts w
    \item $\epsilon$ bezeichnet das leere Wort mit Länge 0, $|\epsilon|=0$
    \item $|w|_{a}$ bezeichnet die Anzahl der Vorkommen des Zeichens a in w
    \item $\sum^{*}$ bezeichnet die Menge aller Wörter, die mit Zeichen von $\sum$ gebildet werden können
\end{itemize}
\subsection{Konkatenation}
Die Konkatenation (Aneinanderhängen) von zwei Wörter wird als $v \cdot w$ ($vw$) notiert.
\begin{itemize}
    \item \textbf{Assoziativität}:
    \\$u \cdot (v \cdot w) = (u \cdot v) \cdot w$
    \item $\epsilon$ ist \textbf{neutrales Element}:
    \\$\epsilon \cdot w = w \cdot \epsilon = w$
    \item Addition der Längen:
    \\$|v \cdot w| = |v| + |w|$
    \item \textbf{n-fache Konkatenation}:
    \\$w^{0} = \epsilon$
    \\$w^{n} = w \cdot w^{n-1}$ für $n > 0$
    \\$|w^{n}| = n \cdot |w|$
\end{itemize}
\section{Sprachen}
Die \textbf{Syntax} einer Sprache beschreibt die Regeln wie "Äußerungen" der Sprache gebildet werden können.
\\Die \textbf{Semantik} beschreibt die Bedeutung der formulierbaren Äußerung.
\\Die \textbf{Pragmatik} beschäftigt sich mit der Nutzung der Sprache.
\\Eine Sprache L über dem Alphabet $\sum$ ist eine Menge von Wörtern über $\sum$ \\
\includegraphics[width=1\textwidth]{"graphics/sprache"}
\subsection{Operationen für Sprachen}
\textbf{Definition}
\begin{itemize}
    \item Die \textbf{Konkatenation $L_{1} \cdot L_{2}$} zweier Sprachen $L_{1}$ und $L_{2}$ ist
    \\$L_{1} \cdot L_{2} = {vw | v \in L_{1}, w \in L_{2}}$
    \item Für eine Sprache L ist $L^{n}$ wie folgt definiert:
    \\$L^{0} = {\epsilon}$
    \\$L^{n} = L \cdot L^{n-1}$, für $n>0$
    \item Die \textbf{Kleene'sche Hülle} $L^{*}$ einer Sprache L ist definiert durchlaufen
    \\$L^{*} = L^{0} \cup L^{1} \cup L^{2} \cup ...$
\end{itemize}
\section{Exkurs: XML}
\textbf{Serialisierung} = Abbildung strukturierter Daten in eiene sequentielle Darstellungsform
\begin{itemize}
    \item \textbf{XHTML}: Beschreibung von Web-Seiten
    \item \textbf{SVG}: zweidimensionale Vektorgrafik
    \item \textbf{ODF}: genormtes Austauschformat für Bürodokumente
    \item \textbf{MathML}: Dokumentenformat zur Darstellung mathematischer Formeln
    \item \textbf{MusicXML}: offenes Dateiformat zum Austausch von Musiknoten
    \item \textbf{WSDL}: Schnittstellen-Beschreibung für Web-Services
\end{itemize}
\subsection{Wohlgeformtheit}
Ein Text ist ein wohlgeformter XML-Text, wenn er folgende Regeln erfüllt:
\begin{itemize}
    \item Ein XML-Text besteht aus genau einem XML-Element
    \item Ein XML-Element beginnt mit einem Anfangstag <tag> und endet mit dem gleichnamigen Endtag </tag>
    \item Elementare Texte können beliebige Zeichenfolgen sein, die aber keine Tags enthalten
\end{itemize}
\subsection{Weitere XML-Details}
\includegraphics[width=1\textwidth]{"graphics/XML-tag"}
\pagebreak
\chapter{Reguläre Ausdrücke}
\section{Syntax reguläre Ausdrücke}
Die Menge der \textbf{regulären Ausdrücke} über einem Alphabet $\epsilon$ ist rekursiv so definiert:
\\Folgendes sind elementare reguläre Ausdrücke:
\begin{itemize}
    \item [(1)] $\emptyset$
    \item [(2)] $\epsilon$
    \item [(3)] \textbf{a} für jedes Zeichen $a \in \epsilon$
\end{itemize}
Sind R und S beliebige reguläre Ausdrücke, dann sind auch folgendes reguläre Ausdrücke:
\begin{itemize}
    \item [(4)] $R \cdot S$
    \item [(5)] $R | S$
    \item [(6)] $R^{*}$
    \item [(7)] $(R)$
\end{itemize}
\textbf{Notationskonvention}
\begin{itemize}
    \item Operator $^{*}$ bindet stärker als Operator $\cdot$
    \item Operator $\cdot$ bindet stärker als Operator $|$
\end{itemize}
\section{Semantik reguläre Ausdrücke}
Die von einem regulären Ausdruck R dargestellte Sprache L(R) ist folgendermaßen definiert:
\begin{itemize}
    \item [(1)] $L(\emptyset) = {}$ \hspace{1,6cm} // leere Sprache, enthält gar kein Wort
    \item [(2)] $L(\epsilon) = {\epsilon}$ \hspace{1,4cm} // Sprache, die nur das leere Wort enthält
    \item [(3)] $L(a) = {a}$ \hspace{1,3cm} // Sprache, die nur das Wort aus dem einzelnen Zeichen a enthält
    \item [(4)] Ist $L(R)=L_{1}$ und $L(S)=L_{2}$ dann ist $L(R \cdot S)=L_{1} \cdot L_{2}$
    \item [(5)] Ist $L(R)=L_{1}$ und $L(S)=L_{2}$ dann ist $L(R | S)=L_{1} \cup L_{2}$
    \item [(6)] Ist $L(R)=L_{1}$ dann ist $L(R^{*})=L_{1}^{*}$
    \item [(7)] Ist $L(R)=L_{1}$ dann ist $L((R))=L_{1}$
\end{itemize}
Die Bedeutung zusammengesetzter Ausdrücke kann so erklärt werden:
\begin{itemize}
    \item \textbf{$R|S$} ist die Menge aller Wörter, die in der Sprache R oder in der Sprache S enthalten sind.
    Sprechweise: "R oder S"
    \item \textbf{RS} ist die Menge aller Wörter, so dass der erste Teil des Worts in der Sprache von R liegt und der Rest in S
    \item \textbf{$R^{*}$} ist die Menge aller Wörter $w=u_{1}u_{2}...u_{n}$ die sich aus bel. vielen Teilwörtern zusammensetzen
\end{itemize}
\textbf{$R^{+}$} ist die Abkürzung für \textbf{$RR^{*}$}
\textbf{$R?$} ist die Abkürzung für \textbf{$R|\epsilon$}
\section{Äquivalenz regulärer Ausdrücke}
Zwei reguläre Ausdrücke R und S heißen äquivalent, falls $L(R) = L(S)$, d.h. wenn beide Ausdrücke die gleiche Sprache darstellen.
\\\textbf{Eigenschaften}
\\Für beliebige reguläre Ausdrücke R, S und T gilt:
\begin{itemize}
    \item $R|S = S|R$
    \item $(R|S)|T = R|(S|T)$
    \item $RS|RT = R(S|T)$
    \item $RT|ST = (R|S)T$
    \item $R|R = R$
    \item $R| \emptyset = \emptyset|R = R$
    \item $(RS)T = R(ST)$
    \item $R \epsilon = \epsilon R = R$
    \item $R \emptyset = \emptyset R = \emptyset$
    \item $R^{**} = R^{*}$
    \item $R^{*} = \epsilon | RR^{*}$ bzw. $R^{*} = \epsilon | R^{+}$
    \item $R^{*} = \epsilon | R^{*}$
\end{itemize}
\section{Anwendung regulärer Ausdrücke}
\begin{tabularx}{\textwidth}{p{0.25\textwidth}|l|X|l}
    \multirow{3}{*}{\textbf{boolean} }
            & \textbf{matches (regex)}\\
            & prüft, ob der String entsprechend dem regulären Ausdruck regex\\
            & aufgebaut ist.\\
    \hline
    \multirow{2}{*}{String}
            & \textbf{replaceAll (String regex, String replacement)}\\
            & ersetzt alle Vorkommen des Musters regex durch replacement\\
    \hline
    \multirow{2}{*}{String}
            & \textbf{replaceFirst (String regex, String replacement)}\\
            & ersetzt das erste Vorkommen des Musters regex durch replacement\\
    \hline
    \multirow{2}{*}{String []}
            & \textbf{split} (String regex, \textbf{int limit})\\
            & Splits this string around matches of the given regular expression\\
\end{tabularx}
\pagebreak
\chapter{Kontextfreie Grammatiken}
\end{document}